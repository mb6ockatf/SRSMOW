\documentclass{article}
\usepackage{hyperref,gensymb,amsmath,graphicx,svg}
% \usepackage{graphicx,amssymb,amsmath,lmodern,wrapfig,csquotes}% savetrees,
	% hyperref,multicol,bookmark,amsthm,thmtools,mathtools,array,
	% tkz-euclide,gensymb,twemojis,lettrine}
% \usepackage[lighttt]{lmodern}
% \usepackage[fontsize=10pt]{fontsize}
% \usepackage[parfill]{parskip}
% \usepackage[T2A]{fontenc}
% \usepackage[utf8]{inputenc}
% \usepackage[russian,english]{babel}
% \DeclareGraphicsRule{.1}{mps}{*}{}
% \DeclareGraphicsRule{.2}{mps}{*}{}

% \renewcommand{\chaptername}{Глава}
% \addto\captionsenglish{\renewcommand{\chaptername}{Глава}}
% \addto\captionsenglish{\renewcommand{\contentsname}{Содержание}}
% \addto\captionsenglish{\renewcommand{\figurename}{Рисунок}}

% \theoremstyle{definition}
% \newtheorem{definition}{Определение}
% \theoremstyle{corollary}
% \newtheorem{corollary}{Следствие}
% \theoremstyle{theorem}
% \newtheorem{theorem}{Теорема}
% \renewcommand{\proofname}{Доказательство}
% \newtheorem*{remark}{Remark}
% \newtheorem*{example}{Пример}
% \newtheorem{definition}{Определение}[section]
% \setlength{\parindent}{0pt}
% \usepackage[a5paper]{geometry}
% \geometry{a4paper, left=6mm, right=6mm, bottom=15mm, top=6mm}
%\geometry{a4paper, margin=0in, bottom=15mm, top=6mm}
% \author{А. Б. Бакушинский, В. К. Власов}

\title{SRSMOW}
\author{@mb6ockatf}
\date{2024}
\graphicspath{{./static/photos/}}
\begin{document}
\maketitle
\tableofcontents
\section{Planets \& The Sun}
\subsection{The Sun}
$\odot$ symbol is often used in relation to the Sun, like $M_\odot, R_\odot,
L_\odot$ -- Solar mass, radius, luminosity.
\label{star:sun}

\subsection{Mercury}
\label{planet:mercury}
\subsubsection{Position}
\begin{itemize}
	\item \hyperref[planet:mercury]{Mercury} is the closest planet to
		\hyperref[star:sun]{the Sun} and the smallest
	\item Spinning around \hyperref[star:sun]{the Sun} takes only 87.97
		\hyperref[planet:earth]{Earth} days -- the shortest spinning
		time in the Solar System
	\item \hyperref[planet:mercury]{Mercury} is an
		\href{https://en.m.wikipedia.org/wiki/Inferior_and_superior_planets}{inferior}
		(interior) planet to \hyperref[planet:earth]{the Earth}
	\item Planet can usually be seen in dusk or dawn, in
		\href{https://en.wikipedia.org/wiki/Twilight}{twilight}. It
		happens due to its proximity to \hyperref[star:sun]{the Sun}
\end{itemize}
\subsubsection{History}
The planet was named after the
\href{https://en.wikipedia.org/wiki/Mercury_(mythology)}{Roman god Mercurius}.
He was the god of commerce and messenger of gods, mediator between gods and
mortals. In Greek mythology he is kind of equal to Hermes
\subsubsection{Surface}
It is similar to \hyperref[planet:earth:moon]{the Moon}: very cratered
\href{https://en.wikipedia.org/wiki/Terrestrial_planet}{terrestrial} planet. No
geological activity for billions of years.
\paragraph{Atmosphere}
\hyperref[planet:mercury]{Mercury} has no atmosphere, because its gravity is too
weak to retain any gases. But there is an exosphere of
\href{https://en.wikipedia.org/wiki/Hydrogen}{hydrogen},
\href{https://en.wikipedia.org/wiki/Helium}{helium},
\href{https://en.wikipedia.org/wiki/Oxygen}{oxygen},
\href{https://en.wikipedia.org/wiki/Sodium}{sodium},
\href{https://en.wikipedia.org/wiki/Calcium}{calcium},
\href{https://en.wikipedia.org/wiki/Potassium}{potassium} and others. Exosphere
is chemically unstable. \href{https://en.wikipedia.org/wiki/Helium}{Helium} and
\href{https://en.wikipedia.org/wiki/Hydrogen}{hydrogen} come with Solar wind.
Generally, the whole chemical composition is affected by Solar wind.
\paragraph{Chemical Compounds}
\hyperref[planet:mercury:spacecraft]{Messenger} found high proportions of
\href{https://en.wikipedia.org/wiki/Calcium}{calcium},
\href{https://en.wikipedia.org/wiki/Helium}{helium},
\href{https://en.wikipedia.org/wiki/Hydroxide}{hydroxide},
\href{https://en.wikipedia.org/wiki/Magnesium}{magnesium},
\href{https://en.wikipedia.org/wiki/Oxygen}{oxygen},
\href{https://en.wikipedia.org/wiki/Potassium}{potassium},
\href{https://en.wikipedia.org/wiki/Silicon}{silicon} \&
\href{https://en.wikipedia.org/wiki/Sodium}{sodium}. These elements either are
brought by Solar wind, or are
\href{https://en.wikipedia.org/wiki/Silicon}{sputtered} from planet's ground
due to Solar wind's powers. Water vapor is present due to comets that strike its
surface.

Water vapor appears out of
\href{https://en.wikipedia.org/wili/Hydrogen}{hydrogen} from Solar wind and
oxygen from rock, and sublimation from reservoirs of water ice in permanently
shadowed polar craters. High amounts of water-related ions like $O^+$, $OH^-$,
\href{https://en.wikipedia.org/wiki/Hydronium}{$H_3O^+$} were detected.

\paragraph{Weather}
\begin{itemize}
	\item Surface is incredibly heated: 100 K (-173\degree C; -280\degree F)
		at night \& up to 700K (427\degree C; 800\degree F) during day
		across equatorial regions
	\item Polar regions are constantly below 180K (-93\degree C; -136\degree
		F)
\end{itemize}
Has no natural satellites

\href{https://en.wikipedia.org/wiki/Very_Large_Array}{VLA} and
\href{https://en.wikipedia.org/wiki/Goldstone_Solar_System_Radar}{GSSR}
researches confirmed that there is ice on the polars of Mercury. It is possible
because deep craters are never really exposed to sunlight. So, they become a
cold trap, where ice accumulate. There is approximately $10^{15}$ kg of ice on
polars.
\subsubsection{Physical Properties}
\paragraph{Basic Properties}
\begin{itemize}
	\item Diameter of 4880 km (like USA)
\end{itemize}
The planet has 3 main layers: core (85\% of all size),
\href{https://education.nationalgeographic.org/resource/mantle/}{mantle},
\href{https://education.nationalgeographic.org/resource/crust/}{crust}. Like
\hyperref[planet:earth]{the Earth}, but our planet's core takes only 55\% of all
it's size

No tectonic plates

Iron core is slowly cooling, causing the whole planet to shrink. Mercury has
already shrinked for 4.4 miles
\paragraph{Magnetic Field}
Mercury has extremely
\href{https://en.wikipedia.org/wiki/Mercury\%27s_magnetic_field}{powerful
magnetic field}. During \hyperref[planet:mercury:splacecreft:mariner10]{Mariner
10} trip in 1973 the fact that \hyperref[planet:earth]{earth's} magnetic fielt
is 1.1\% of Mercury's one. Actually, it equals to 300nT (nanoTesla) on equator.
Existence of such a magnetic field is explained by planet's liquid iron-rich
core.

\paragraph{Orbit \& Rotation}

Mercury's axis has the smallest tilt of all Solar System's planets (about
$\frac{1}{30}\degree$.)
\subparagraph{Orbital Eccentritet}
Orbital eccentritet is largest of all known objects in Solar System. At
perihelion, its distance to \hyperref[star:sun]{the Sun} is 66\% of one in
aphelion. Distance from \hyperref[star:sun]{the Sun} ranges from $4\times 10^6$
to $70\times 10^6$ km. Complex variations of surface temperature are possible
due to such an orbit.
\subsubsection{Spacecrafts}
\label{planet:mercury:spacecraft}
\begin{itemize}
	\item 2 spacecrafts have visited \hyperref[planet:mercury]{Mercury}
		\begin{itemize}
			\item \href{https://solarsystem.nasa.gov/missions/mariner-10/in-depth/}{Mariner 10
				(NASA)} in 1974 and 1975
				\label{planet:mercury:spacecraft:mariner10}
				\begin{itemize}
					\item $1^{\text{st}}$ spacecraft to
						explore
						\hyperref[planet:mercury]{Mercury}
					\item $1^{\text{st}}$ spacecraft to use
						gravity of one planet to reach
						another
					\item $1^{\text{st}}$ spacecraft to
						return data on a login-period
						comet
					\item $1^{\text{st}}$ spacecraft to
						explore 2 planets in 1 mission
					\item $1^{\text{st}}$ spacecraft to use
						gravity assistance to change its
						direction
					\item $1^{st}$ spacecraft to return to
						its target after initial
						encounter
					\item $1^{\text{st}}$ probe to use Solar
						wind as a major mean of
						spacecraft orientation during
						flight
				\end{itemize}
			\item
				\href{https://solarsystem.nasa.gov/missions/messenger/in-depth/}{Messenger
				(NASA)}, launched in 2004, orbited around
				Mercury for over 4000 times in 4 years before
				running out of energy and crashong into the
				planet's surface on 30 April, 2015.
				\begin{itemize}
					\item Explored surface ingredients,
						revealed geological history,
						discobered details about its
						internal magnetic field,
						verified polar water-ice
						existance.
				\end{itemize}

		\end{itemize}
	\item Bepicolombo spacecraft is planned to arrive to
		\hyperref[planet:mercury]{Mercury} in 2015
\end{itemize}
\subsubsection{Other}
\begin{center}
	\begin{tabular}{l l}
		crates & famous humanitarians\\
		chains of crates & radioobservatories\\
		scarps & explorators\\
		furrows & architecture (buildings, etc)\\
		valleys & abandoned old settlements (Angkor)\\
	\end{tabular}
\end{center}

\subsection{Venus}
\label{planet:venus}
\subsubsection{Position}
\begin{itemize}
	\item \hyperref[planet:venus]{Venus} is the $2^{\text{nd}}$ closest
		planet to \hyperref[star:sun]{the Sun} and the $3^{\text{rd}}$
		smallest
	\item \hyperref[planet:venus]{Venus} is an
		\href{https://en.m.wikipedia.org/wiki/Inferior_and_superior_planets}{inferior}
		(interior) planet to \hyperref[planet:earth]{the Earth}
	\item Has no moons, like \hyperref[planet:mercury]{Mercury}.
	\item Usually can be seen in dusk or dawn, because of its inferiority:
		it is always close to \hyperref[star:sun]{the Sun}.
\end{itemize}
\subsubsection{History}
The planet was named after the
\href{https://en.wikipedia.org/wiki/Goddess_Venus}{Roman goddess Venus}.
\subsubsection{Surface}
70\% is smooth, volcanic plains. Atmosphere is dense, contains Sulfur
substances. Surface is therefore completely consealed for outer observer by
clouds. Consists of 2 highland continents: southern (Aphrodite Terra) \&
northern (Ishtar Terra).

Is reported to have plate tectonics. The surface in such formed due to past
volcanic activities, where average diameter of a volcano was 100 km.

Atmosphere is 96.5\% $CO_2$.

\subsubsection{Physical Properties}
\paragraph{Basic Properties}
\begin{itemize}
	\item Diameter of 12103.6 km
\end{itemize}

Atmosphere of Venus is something worth being mentioned. It produces the most
powerful greenhouse effect in the Solar System, and the surface
temperature of the planet is higher than the one of
\hyperref[planet:mercury]{Mercury}. Minimal temperature on Mercury is 53K,
maximal -- 700K. Minimal temperature on Venus is 735K.

\paragraph{Magnetic Field}
Much weaker than one of \hyperref[planet:earth]{the Earth}.
\paragraph{Orbit \& Rotation}
Rotates around itself in the 243 Earth days -- slowest rotaion of all Solar
System planets. Venusian day lasts longer than Venusian year.

\subparagraph{Orbital Exxentritet}
Eccentricity is less than 0.01, smallest among all planets of Solar System.
\subsubsection{Spacecrafts}
A bunch of USSR Venera spacecrafts, japanese IKAROS

\subsection{The Earth}
\label{planet:earth}
\subsubsection{Position}
\subsubsection{History}
\subsubsection{Surface}
\subsubsection{Physical Properties}
\paragraph{Basic Properties}
\paragraph{Magnetic Field}
\paragraph{Orbit \& Rotation}
\subparagraph{Orbital Exxentritet}
\subsubsection{Spacecrafts}
\subsubsection{Other}
\subsubsection{The Moon}
\label{planet:earth:moon}

\subsection{Mars}
\subsubsection{Position}
\subsubsection{History}
\subsubsection{Surface}
\subsubsection{Physical Properties}
\paragraph{Basic Properties}
\paragraph{Magnetic Field}
\paragraph{Orbit \& Rotation}
\subparagraph{Orbital Exxentritet}
\subsubsection{Spacecrafts}
\subsubsection{Other}

\subsection{Jupiter}
\subsubsection{Position}
\subsubsection{History}
\subsubsection{Surface}
\subsubsection{Physical Properties}
\paragraph{Basic Properties}
\paragraph{Magnetic Field}
\paragraph{Orbit \& Rotation}
\subparagraph{Orbital Exxentritet}
\subsubsection{Spacecrafts}
\subsubsection{Other}

\subsection{Saturn}
\subsubsection{Position}
\subsubsection{History}
\subsubsection{Surface}
\subsubsection{Physical Properties}
\paragraph{Basic Properties}
\paragraph{Magnetic Field}
\paragraph{Orbit \& Rotation}
\subparagraph{Orbital Exxentritet}
\subsubsection{Spacecrafts}
\subsubsection{Other}

\subsection{Uran}
\subsubsection{Position}
\subsubsection{History}
\subsubsection{Surface}
\subsubsection{Physical Properties}
\paragraph{Basic Properties}
\paragraph{Magnetic Field}
\paragraph{Orbit \& Rotation}
\subparagraph{Orbital Exxentritet}
\subsubsection{Spacecrafts}
\subsubsection{Other}
\subsection{Neptune}

\subsubsection{Position}
\subsubsection{History}
\subsubsection{Surface}
\subsubsection{Physical Properties}
\paragraph{Basic Properties}
\paragraph{Magnetic Field}
\paragraph{Orbit \& Rotation}
\subparagraph{Orbital Exxentritet}
\subsubsection{Spacecrafts}
\subsubsection{Other}

\section{Constellations}
\ldots list sorted by area in square degrees
\subsection{Northern Hemisphere}
\subsubsection{Ursa Major}
northern
\subsubsection{Hercules}
northern
\subsubsection{Pegasus}
northern
\subsubsection{Draco}
northern
\subsubsection{Leo}
northern
\subsubsection{Bootes}
northern
\subsubsection{Pisces}
northern
\subsubsection{Sagittarius}
northern
\subsubsection{Cygnus}
northern
\subsubsection{Taurus}
northern
\subsubsection{Camelopardalis}
northern
\subsubsection{Andromeda}
northern
\subsubsection{Auriga}
northern
\subsubsection{Perseus}
northern
\subsubsection{Cassiopeia}
northern
\subsubsection{Cepheus}
northern
\subsubsection{Lynx}
northern
\subsubsection{Libra}
\subsubsection{Gemini}
northern
\subsubsection{Cancer}
northern
\subsubsection{Canes Venatici}
northern
\subsubsection{Aries}
northern
\subsubsection{Coma Berenices}
northern
\subsubsection{Lyra}
northern
\subsubsection{Ursa Minor}
northern
\subsubsection{Leo Minor}
northern
\subsubsection{Lacerta}
northern
\subsubsection{Delphinus}
northern
\subsubsection{Corona Borealis}
northern
\subsubsection{Triangulum}
northern
\subsubsection{Sagitta}
northern
\subsubsection{Equulus}
northern

\subsection{Southern Hemisphere}
\subsubsection{Hydra}
southern
\subsubsection{Virgo}
equatorial
\subsubsection{Cetus}
equatorial
\subsubsection{Eridanus}
southern
\subsubsection{Centaurus}
southern
\subsubsection{Aquarius}
southern
\subsubsection{Ophiuchis}
equatorial
\subsubsection{Puppis}
southern
\subsubsection{Orion}
equatorial
\subsubsection{Vela}
southern
\subsubsection{Scorpius}
southern
\subsubsection{Carina}
southern
\subsubsection{Monoceros}
equatorial
\subsubsection{Sculptor}
southern
\subsubsection{Phoenix}
southern
\subsubsection{Capricornus}
southern
\subsubsection{Fornax}
southern
\subsubsection{Canis Major}
southern
\subsubsection{Pavo}
southern
\subsubsection{Grus}
southern
\subsubsection{Lupus}
southern
\subsubsection{Sextans}
equatorial
\subsubsection{Tucana}
southern
\subsubsection{Indus}
southern
\subsubsection{Octans}
southern
\subsubsection{Lepus}
southern
\subsubsection{Crater}
southern
\subsubsection{Columba}
southern
\subsubsection{Vulpecula}
southern
\subsubsection{Telescopium}
southern
\subsubsection{Horologium}
southern
\subsubsection{Pictor}
southern
\subsubsection{Piscis Austrinus}
southern
\subsubsection{Hydrus}
southern
\subsubsection{Antlia}
southern
\subsubsection{Ara}
southern
\subsubsection{Pyxis}
southern
\subsubsection{Microscopium}
southern
\subsubsection{Apus}
southern
\subsubsection{Corvus}
southern
\subsubsection{Canis Major}
equatorial
\subsubsection{Dorado}
southern
\subsubsection{Norma}
southern
\subsubsection{Mensa}
southern
\subsubsection{Volans}
southern
\subsubsection{Musca}
southern
\subsubsection{Chamaeleon}
southern
\subsubsection{Corona Australis}
southern
\subsubsection{Caelum}
southern
\subsubsection{Reticulum}
southern
\subsubsection{Triangulum Australe}
southern
\subsubsection{Scutum}
southern
\subsubsection{Circinus}
southern
\subsubsection{Crux}
southern

\section{Galaxies}
\end{document}
